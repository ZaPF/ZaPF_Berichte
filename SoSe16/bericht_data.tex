Vom 04. bis 08. Mai 2016 fand in Konstanz die 74. Zusammenkunft
aller deutschsprachigen Physik-Fachschaften (kurz: ZaPF) statt.  Die ZaPF ist
die Bundesfachschaftentagung der Physik und versteht sich dabei als eine
grundlegende Basis zum Austausch zwischen den Physik-Fachschaften im
deutschsprachigen Raum über hochschulpolitische Themen und darüber hinaus als
Gremium der Meinungsbildung und -äußerung der Physikstudierenden. Sie tagt
einmal pro Semester an unterschiedlichen Hochschulen, wobei sie von der
Physik-Fachschaft der ausrichtenden Hochschule selbst organisiert wird. \\

Diese Aufgabe wurde im Sommersemester 2016 von der Fachschaft Physik der Universtiät Konstanz übernommen. 
Es nahmen 229 Fachschaftler*innen aus insgesamt 49 teilnehmenden Fachschaften teil, 
die sich in über  48 Arbeitskreisen austauschten und Positionen zu verschiedenen Themen erarbeiteten.

Schwerpunkte der ZaPF in Konstanz waren die Auseinandersetzung mit dem CHE-Ranking, 
die Kritik an Zweiklassenstudiensystemen, Zivilklauseln an Hochschulen, den Umgang mit 
Drittmitteln in der Forschung, die dritte Runde der Exzellenzinitiative, Quotenregelungen in Hochschulgremien, 
die internationale Parallelschaltung von Semesterzeiten, Fachdidaktik im Lehramtsstudium, 
Zulassungsbeschränkungen, die Akkreditierung von Hochschulen und Studiengängen, die Novellierung des Wissenschaftzeitvertragsgesetzes
(WissZeitVG), Programmierkenntnisse im Physikstudium und die Vernetzung mit Doktorandenvertretungen.

\newpage

\section*{CHE}  
Bereits seit 2007 beschäftigt sich die ZaPF kontinuierlich mit dem CHE-Ranking und dessen
Veröffentlichungen sowie Methoden. Nach der Ablehnung des CHE-Rankings 2013, konnte
in Zusammenarbeit mit der KFP und der jDPG 2014 wieder konstruktiv mit dem CHE
zusammengearbeitet werden. Dabei wurde insbesondere über die momentane Berichterstattung und 
Veröffentlichung des CHE-Rankings geredet. Im Zuge dessen wurde das von der ZaPF 
auf der letzten Tagung erstellte Positionspapier überarbeitet, sodass es nun von der 
ZaPF und dem CHE gemeinsam getragen wird. Außerdem wird das CHE andere Resolutionen 
der ZaPF zum CHE-Ranking leichter zugänglich machen und es soll unsere Homepage vom CHE 
verlinkt werden, damit sich Interessierte besser ein Bild von unserer Meinung machen können.

\section*{Zweiklassenstudiensysteme}
Noch im Laufe diesen Jahres sollen an Universitäten in Thüringen wieder Studiengänge mit 
dem Diplom als Abschluss eingeführt werden. Der Arbeitskreis beschäftigte sich mit der Situation,
diskutierte diese und verfasste als Folge eine Resolution zu „Zweiklassenstudiensystemen“, 
in der unter anderem die überhastete Neugestaltung von Diplomstudiengängen in Thüringen und insbesondere Ilmenau kritisiert wird.

\section*{Zivilklausel}
In diesem Folge-Arbeitskreis wurde anhand einer Vorlage eine Resolution diskutiert und
ausgearbeitet. Die Resolution soll nicht als direkte Kritik sondern als Denk- und Diskussionsanstoß 
für bereits kritisch eingestellte Leute dienen. Nach längerer Diskussion über die Formulierungen der Resolution 
im Endplenum wurde sie zur genauen Ausarbeitung auf die nächste ZaPF vertagt.

\section*{Veröffentlichungspflicht von Drittmittelergebnissen}
In dem Arbeitskreis „Veröffentlichungspflicht von Drittmittelforschung“ bearbeitet 
und diskutiert die ZaPF seit dem Sommersemester 2014 Fragestellungen, zu Themen wie 
Veröffentlichungspflicht und Transparenz von drittmittelfinanzierter Forschung. 
Unter Berücksichtigung der Ergebnisse der vorangegangenen Arbeitskreise vergangener 
ZaPFen wurde nun auf Grundlage eines Resolutionsentwurfes eine abschließende Stellungnahme 
entwickelt und verabschiedet. %(LINK ZU RESO)
Die ZaPF kam zu dem Schluss, eine Veröffentlichtungspflicht von Forschungsergebnissen zu fordern, 
da auch bei drittmittelfinanzierter Forschung öffentliche Güter genutzt werden. Diese sollen in leicht zugänglicher Form der interessierten Öffentlichkeit zur Verfügung gestellt werden. Auf die Veröffentlichung von Nullergebnissen ist in einem weiteren 
Arbeitskreis eingegangen worden. In diesem wurde ein Grundstein für eine Stellungnahme gelegt, 
welche auf der kommenden ZaPF weiter ausgearbeitet werden soll.



\section*{Exzellenzinitiative}
Am 16. Juni wurden von der GWK (Gemeinsame Wissenschaftskonferenz) die Förderrichtlinien 
zur dritten Runde der Exzellenzinitiative verabschiedet. Diesen Beschluss und den im Januar 2016 veröffentlichten \href{http://www.gwk-bonn.de/fileadmin/Papers/Imboden-Bericht-2016.pdf}{Imboden-Bericht} hat 
die ZaPF als Anlass genommen, sich zum Einfluss der Exzellenzinitative auf die Lehre und Forschung zu äußern. \\
In dem Arbeitskreis wurde im Anschluss an eine Diskussion eine Stellungnahme verfasst, in der die ZaPF 
ihre Kritikpunkte bezüglich der negativen Auswirkung auf die Qualität der Lehre, sowie wissenschaftliche 
Koorporationen und einer flächendeckenden Ausfinanzierung der Hochschulen festgehalten hat. Darüberhinaus fordert 
sie die Fachschaften aller Fachrichtungen dazu auf, sich aktiv in die Debatten über die Bewerbung ihrer Hochschulen einzubringen. 

\section*{Frauenquote}

Auf dieser ZaPF wurde wieder über Frauenquoten diskutiert. Im Ergebnis hat sich die ZaPF 
gegen die Ungleichbehandlung verschiedener Statusgruppen bei der Besetzung von universitären 
Gremien ausgesprochen, so wie es z.B. in Nordrhein Westfalen der Fall ist. 

Dort gilt eine allgemeine Frauenquote in universitären Gremien für alle Statusgruppen, die 
Professorenschaft darf jedoch bei Nichterreichen der Quote ihren Anteil des Gremiums 
auch nur nach dem tatsächlichen Geschlechterverhältnis in der Professorenschaft besetzen. 
Die Teilnehmenden der ZaPF konnten sich darauf einigen, dass diese Ungleichbehandlung nicht statthaft ist.

\section*{Internationale Semesterzeiten}
Die Vorlesungszeiten variieren im europäischen Ausland stark. In Deutschland endet die Vorlesungszeit im 
Wintersemester zu Zeiten an denen andere europäische Länder bereits mit den Vorlesungen im Sommer-/Frühlingssemester beginnen. \\
In 2011 gab es bereits einen AK zu dem Thema, dessen Resolution vertagt und dann nicht 
wieder aufgegriffen wurde. Deshalb wurde in Konstanz ein Positionspapier zu Internationalen Semesterzeiten verabschiedet.

\section*{Lehramt Physik}
Auch in diesem Semester hat sich die ZaPF mit Themen des Lehramtes beschäftigt. 
Auf der Grundlage von Stellungnahmen des Nationalen MINT-Forums, der DPG un der 
Expertenkommission des Landes Nordrhein-Westfalen hat sie ein Positionspapier verfasst, 
dass sich mit fachdidaktischen Planstellen und der fachdidaktischen Ausbildung in 
der ersten und zweiten Ausbildungsphase beschäftigt. Des weiteren befürwortet die ZaPF 
fachdidaktische Summer Schools und Kolloquien. \\
Um auf  anfallende Themen bezüglich des Lehramtes zeitnah reagieren zu können, wurde ein ständiger Arbeitskreis Lehramt eingerichtet. 


\section*{Zulassungsbeschränkungen}
In diesem AK ging es um Zugangsbeschränkungen für Bachelorstudiengänge, 
die über ein Abitur hinaus gehen, sowie um Zulassungsbeschränkungen für Masterstudiengänge.
Es wurde sich in einem Positionspapier gegen Zulassungsbeschränkungen wie den 
NC für Bachelor- oder Grenznoten für Master-Studiengänge ausgesprochen. 
So soll die Konkurrenz zwischen Schüler*innen bzw. Studierenden vermieden werden. 
Zudem wurde eine klare Kennzeichnung der inhaltlichen Anforderungen der Master-Studiengänge gefordert. 
Auf der nächste ZaPF soll auf Basis des Positionspapieres eine Resolution erarbeitet werden. 
Außerdem wurde die Problematik internationaler Studierenden im Master  diskutiert, 
welche mit falschen Voraussetzungen und/oder falschen Erwartungen kommen. \\


\section*{Akkreditierung}
Die ZaPF hat ein Positionspapier zum deutschen Akkreditierungssystem beschlossen, 
indem sie sich für ein gutachter*innenzentriertes Verfahren zur Qualitätsprüfung und -sicherung 
von Studiengängen ausspricht. Dabei betont die ZaPF die Wichtigkeit der Gutachtergruppe sowohl 
was ihre Zusammensetzung, wie auch ihre Qualifikation angeht und bewertet dies als Stärke 
des deutschen Akkreditierungssystems. Die ZaPF kritisiert den Preisdruck und den Konkurrenzdruck 
zwischen den Agenturen sowie den zu großen Entscheidungsspielraum der unterschiedlichen Agenturen, 
unter welchem die Vergleichbarkeit der Verfahren leidet. Ein weiterer Kritikpunkt 
ist, dass eine Befangenheit gegenüber bestimmten Hochschulen nicht ausgeschlossen werden kann. 
Alle diese Kritikpunkte resultieren im Wesentlichen aus dem offenen Wettbwerb zwischen 
den Agenturen und die ZaPF strebt eine entsprechende Veränderung des Akkreditierungsystems an.

\section*{Wissenschaftszeitvertragsgesetz (WissZeitVG)}

Das Wissenschaftszeitsvertragsgesetz (WissZeitVG) ist ein Gesetz mit dem schon seit über 
zehn Jahren jeder Physikstudierende spätestens zu Beginn seiner Promotions in Berührung geraten ist. 
Durch die Neufassung vom März diesen Jahres werden studentische Beschäftigte auch direkt angesprochen, 
da eine Höchstbeschäftigungsdauer eingeführt wurde.\\

Im Arbeitskreis auf dieser ZaPF wurde die aktuelle Version des WissZeitVG vorgestellt und 
begonnen Positionen der ZaPF zu arbeiten die auf nächsten ZaPFen weiter bearbeitet werden. 
Auf dieser ZaPF wurde allerdings schon eine Resolution zur Einordnung studentischer Beschäftigungsverhältnisse verabschiedet.\\

Die Neufassung des WissZeitVG sieht eine maximale Beschäftigungsdauer für studentische 
Beschäftigte von sechs Jahren für wissenschaftlichen und künstlerische Hilfstätigkeiten vor. 
Da diese nicht näher definiert sind entsteht eine gefährliche Unklarheit welche Studijobs nach 
WissZeitVG befristet werden dürfen oder ob nicht bestimmte Studijobs nach Teilzeit- und Befristungsgesetz befristet werden müssen.\\

Da die maximale, sachgrundlose Befristungsdauer des Teilzeit- und Befristungsgesetzes mit zwei 
Jahre noch weit unter der Befristungshöchstdauer des WissZeitVG liegt spricht sich die ZaPF 
dafür aus alle studentischen Beschäftigungsverhältnisse als wissenschaftliche oder künstlerische Hilfstätigkeiten anzusehen. \\

\section*{Programmierkenntnisse}
Fast jeder Physikstudent muss heutzutage im Laufe seines Studiums mehrfach programmieren. 
Im AK wurden deshalb die Konzepte verschiedener Unis sowie allgemeine Anforderungen an die 
Vermittlung von guter wissenschaftlicher Praxis beim Programmieren diskutiert. 
Es wurde ein entsprechendes Positionspapier, über die Grundkenntnisse des Programmierens, 
die Physikstudierende besitzen sollten, geschrieben und im Endplenum verabschiedet.

\section*{Doktorand*innenvertretung}
Im Moment gibt es Bestrebungen die  Vertretung der Doktorand*innen, 
welche unter anderem aufgrund der Zuordnung zu verschiedenen Statusgruppen schwierig ist, 
an deutschen Hochschulen zu verbessern.  Die ZaPF hat sich bisher hauptsächlich als Studierendendenvertretung 
betrachtet und die Interessen von Promovierenden bisher ohne Vertretungsanspruch diskutiert. 
Nun spricht sie sich in einem Positionspapier dafür aus auch die Interessen 
und Problematiken von Promovierenden der Physik zu vertreten. 
Hierzu wollen wir Promovierende zur ZaPF einladen.



\section*{Weitere Themen}

Es wurde über die Ursachen der hohen Abbrecherquoten in physikalischen und 
physiknahen Studiengängen diskutiert und einige Lösungsansätze angesprochen.

Da das Abiturwissen besonders in Mathematik teilweise von den Anforderungen 
zum Studienbeginn abweicht, wurde festgestellt, dass Vorkurse im allgemeinen 
über den Abiturstoff hinausgehen sollten, um eventuelle Lücken zu schließen. 
Sie dürfen aber auf keinen Fall Vorraussetzung zum Studium werden.

Nach einer Diskussion über die Kriterien guter Lehre diskutiert, wurde mit dem 
Thema fortgefahren, wie gute Lehre gefördert und garantiert werden kann. 
Hier wurden Möglichkeiten gesammelt, wie Lehrende qualifiziert und motiviert werden können, 
wie Kommunikation über Lehre gefördert werden kann und wie Lehrende ausgewählt werden sollten.

Nach einem Vortrag über Philosophie in der Physik und die Wissenschaftsethik von Dr. Marius
Backmann wurde das Zentrum für Wissenschaftstheorie in Münster
vorgestellt. Anschließend gab es Tipps wie ein Modul Wissenschaftsethik integriert werden kann.

Aufbauend auf der Arbeit der letzten ZaPFen wurden Finanzierungsmöglichkeiten für den 
Studienführer (eine Übersicht der physikalischen und physiknahen Studiengänge in 
Deutschland, Österreich und der Schweiz für Studieninteressierte) diskutiert und dieser aktualisiert und erweitert.

Die Symptompflicht auf Attesten bei Versäumnis von Prüfungen, die an einigen Hochschulen 
angewendet wird, wurde als problematisch erkannt. Es werden weitere Informationen gesammelt um in Dresden zu diesem Thema weiterzuarbeiten.

Über die Vor- und Nachteile der Verschulung außerhalb des Bologna-Prozesses, 
insbesondere über Anwesenheitspflicht, herrschte Uneinigkeit. 
Es wurde eine Stoffsammlung mit dem Ziel einer Stellungnahme in Dresden erstellt.

Eine Liste aller Arbeitskreise und deren Ergebnisse sind im
Reader\footnote{\href{http://www.zapfev.de/reader/Name_des_readers.pdf}{\url{http://www.zapfev.de/reader/Name_des_readers.pdf}}}
zur ZaPF veröffentlicht.

\vfill

Die nächste ZaPF findet vom \emph{10.\ bis 13.\ November 2016} an der  \emph{Technischen Universität Dresden}\footnote{\href{https://zapf.pfsr.de/}{\url{https://zapf.pfsr.de/}}} statt.

Fragen und Anregungen können gerne an den \emph{Ständigen Ausschuss der Physik-Fachschaften}\footnote{\href{mailto:stapf@googlegroups.com}{\url{stapf@googlegroups.com}}} gerichtet werden.

Alle Stellungnahmen der ZaPF und weitere Informationen sind auf \href{http://www.zapfev.de}{\url{www.zapfev.de}} zu finden.

