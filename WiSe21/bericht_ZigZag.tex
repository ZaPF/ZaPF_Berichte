Vom 11.-14. November 2021 fand die 85. ZaPF (Zusammenkunft aller Physik-Fachschaften), die Bundesfachschaftentagung der Physik statt. Die ZaPF versteht sich als meinungsäußerndes Gremium der Physikstudierenden und ermöglicht den Austausch zwischen verschiedenene Fachschaften im deutschsprachigen Raum. Bei dieser einmal im Semester stattfindenden Versammlung werden hochschulpolitische Themen und Aspekte der Fachschaftsarbeit diskutiert. Im Normalfall wird die ZaPF von der Physik-Fachschaft einer Hochschule organisiert und vor Ort ausgerichtet. Dieses Mal fand die ZaPF zum ersten Mal seit dem Sommersemester 2020 wieder teilweise in Präsenz statt und wurde auf mehrere Standorte verteilt, um die Anzahl der Teilnehmenden pro Standort möglichst gering zu halten. Austragungsorte waren Göttingen (zwei getrennte Gruppen) und Köln, außerdem gab es die Möglichkeit, der Veranstaltung auch in digitaler Form beizuwohnen. 

Insgesamt waren an beiden Standorten und in digitaler Form zusammen 44 Fachschaften mit ca. 150 Teilnehmenden vertreten. Es wurden verschiedenste Themen rund um \textbf{das Physikstudium und Gremienarbeit} in mehr als 45 verschiedenen Arbeitskreisen und Workshops behandelt.

Die Schwerpunkte dieser ZaPF lagen dabei vor allem auf folgenden Themen:
\begin{itemize}
\item \glqq BAföG - Notwendigkeit einer Novellierung\grqq : Hierzu wurde eine Resolution an die zu dem Zeitpunkt noch verhandelnden Koalitionspartner verschickt, in der Bezug auf den Forderungskatalog der ZaPF des Wintersemesters 2020 genommen wurde.
\item Zum geplanten Versammlungsgesetz in Nordrhein-Westfalen wurde eine Resolution erarbeitet, die an die Landtagsfraktionen in NRW versandt wurde
\item FOSS
\item MRVO
\item Antrag zur Anerkennung von Leistungspunkten für Bürgerschaftliches Engagement
\end{itemize}

\section*{BAföG}
Das Thema BAföG ist auf der ZaPF schon häufiger behandelt wurden. Diesmal war das Thema im Bezug auf die Koalitionsgespräche der Bundesregierung aber von besonderer Dringlichkeit und Relevanz. Die ZaPF begrüßt die Reform des BAföGs, vermisste in \textbf{den Sondierungsgesprächen} aber konkrete Änderungsvorschläge. Insbesondere im Hinblick auf Chancengerechtigkeit hofft die ZaPF auf eine zukunftsweisende Regierungsperiode, eine damit verbundene Novellierung des BAföGs und verwies noch einmal auf ihren bereits veröffentlichten Forderungskatalog. 

\section*{Versammlungsgesetz}
Die aktuellen Entwicklungen zum Entwurf des Versammlungsgesetztes in NRW wurden von der ZaPF mit Sorge beobachtet. Die ZaPF hat darum ein Papier mit Kritikpunkten ausgearbeitet\footnote{\url{https://zapfev.de/resolutionen/wise21/Versammlungsgesetz_NRW/VersammlungsgesetzNRW.pdf}} und der Landesregierung zukommen lassen. Darin wird unter anderem kritisiert, dass das geplante Gesetz die Versammlungsfreiheit einschränkt und somit weitreichende Folgen für Proteste von Studierenden, unter anderem zu hochschulpolitischen Themen, hat. So werden Studierende eingeschüchtert und in der Wahrnehmung ihrer Grundrechte gehemmt.

\section*{FOSS}
Die ZaPF unterstützt die Initiative Public Code der Free Software Foundation Europe, unterzeichnet deren offenen Brief, der die Nutzung Freier Open Source Software (FOSS) bei öffentlich finanzierter Software fordert und empfiehlt die Bewerbung dieser Petition\footnote{\url{https://zapfev.de/resolutionen/wise21/FOSS/FOSS.pdf}}. Für eine demokratischere Hochschule braucht es auch demokratische und damit freie Open Source Software, die Partizipation fördert. Die Freiheit der Forschung und Lehre muss konsequent auch in Software fortgesetzt werden. Durch den Einsatz von FOSS würden weiterhin Forschende entlastet werden, wissenschaftlichen Institutionen Geld einsparen und doppelte Arbeit vermieden werden, außerdem können soziale und finanzielle Hürden abgebaut werden. 

\section*{MRVO} 
Die ZaPF befürwortet die Überarbeitung der Musterrechtsverordnung (MRVO) und schließt sich der Kritik der KaWuM (Konferenz aller Werkstofftechnischen und Materialwissenschaftlichen Studiengänge) insbesondere in folgenden Punkten an: Externe Studierende (nach Möglichkeit aus dem studentischen Akkreditierungspool) müssen Bestandteil der Gutachter*innengruppe bleiben. Außerdem darf die Ausgestaltung von hochschulinternen Qualitätsmanagementsystemen nicht ohne externe Expertise und Prüfung stattfinden. Auch die Regelung zur Mindestgröße von 5 ECTS pro Modul darf nicht ersatzlos gestrichen werden. Zusätzlich formuliert die ZaPF eigene, ergänzende Kritikpunkte:\\ Die Streichung der Unterteilung in konsekutive und weiterbildende Studiengänge nimmt dem Gesetzgeber die Möglichkeit, grundsätzliche Unterschiede zwischen diesen Arten der Ausbildung einheitlich festzulegen und wird daher abgelehnt. Die Überprüfung der Umsetzung und konkreten Auswirkungen der Maßnahmen zu Frauenförderung und Geschlechtergerechtigkeit sollten nicht nur auf die Ebene der Systemakkreditierung beschränkt sein, sondern auch im Rahmen interner Akkreditierungsverfahren weiter geprüft werden.

\section*{Anerkennung Bürgerlichen Engagements}
Die ZaPF hat sich auf dieser Tagung tiefergehend mit dem Engagement in akademischen und studentischen Gremien beschäftigt und fordert die Schaffung von Anrechnungsmöglichkeiten für bürgerschaftliches Engagement, insbesondere in Form so genannter “Containermodule” wie von der Hochschulrektorenkonferenz vorgestellt\footnote{\url{https://zapfev.de/resolutionen/wise21/Anrechnung_buergerschaftliches_Engagement/Anrechnung_buergerschaftliches_Engagement.pdf}}. Für eine solche Anrechnungsmöglichkeit sprechen unter anderem folgende Punkte: Zum einen ist es Studierenden, die finanziell auf Nebenjobs oder BAföG angewiesen sind oder Kindern oder pflegebedürftige Verwandte haben, oft nicht oder nur schwer möglich, sich bürgerschaftlich zu engagieren. Sie sind von der Mitarbeit in beispielsweise Hochschulgremien indirekt ausgeschlossen und in diesen nicht ausreichend repräsentiert.\\ Weiterhin erwerben Studierende durch ihr Engagement verschiedenste Schlüsselkompetenzen und bereichern das Leben an Hochschulen durch eigenständig organisierte Projekte (Institutsfeste, Betreuung der Erstsemester, Fahrten, Fachvorträge…). Zuletzt könnten so auch   Nachwuchsprobleme der studentischen Vertretung in universitären Gremien, die auf den hohen Druck das Studium in kürzester Zeit abzuschließen zurückzuführen sind, gelöst werden.

\section*{Sonstiges}
Neben den hier kurz vorgestellten Themen, hat sich die ZaPF auch mit anderen Thematiken auseinander gesetzt, bei denen eine abschließende Positionierung aber noch aussteht. So wird beispielsweise das Thema Nachhaltigkeit sehr intensiv behandelt und auf vergangenen ZaPFen wurden bereits verschiedene Ideen und Konzepte gesammelt, um Hochschulen nachhaltiger zu gestalten. Weiterhin wird an einer Resolution zu diesem Thema gearbeitet, die auf einer der nächsten ZaPFen beschlossen werden soll.\\
Auch das Thema Studienreform ist ein wiederkehrendes Thema der ZaPF. Mit Hilfe eines selbstentwickelten Tools können Strukturdiagramme erstellt werden, um Studiengänge besser analysieren und vergleichen zu können. Dieses Projekt soll auch in Zukunft fortgeführt und um weitere Studiengänge ergänzt werden. \\
Weitere Themen, die regelmäßig auf der ZaPF behandelt werden und auch zukünftig weiter von Bedetung sein werden, sind unter anderem Wissenschaftskommunikation und Open Data. 

Die gesammelten Stellungnahmen der ZaPF können unter www.zapfev.de eingesehen werden. Fragen und Anregungen können gerne an den Ständigen Ausschuss der Physik-Fachschaften (StAPF), das vertretende Gremium der ZaPF, geschickt werden.
Die nächste ZaPF wird von der Fachschaft der Ruhr-Universität Bochum ausgerichtet und findet vom 03.-07. Juni 2022 statt.