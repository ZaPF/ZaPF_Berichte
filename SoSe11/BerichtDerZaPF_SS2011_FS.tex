\documentclass{scrartcl}
\usepackage[utf8]{inputenc}
\usepackage[T1]{fontenc}
\usepackage[ngerman]{babel}
\usepackage{graphicx}

\usepackage{fourier} 			% Schriftart
\usepackage[scaled=0.81]{helvet} 	% Schriftart
\usepackage{setspace}			% Zeilenabstand

\usepackage{url}
% \usepackage{tocloft} 			% Paket für Table of Contents

\usepackage{xcolor}
% \definecolor{tocblue}{HTML}{000033}
\definecolor{urlred}{HTML}{660000}

\usepackage{hyperref}
\hypersetup{
  colorlinks=true,	
  linkcolor=black,	% Farbe der internen Links (u.a. Table of Contents)
  urlcolor=urlred}	% Farbe der url-links

\parindent 0pt 				% Absatzeinrücken verhindern
\parskip 12pt 				% Absätze durch Lücke trennen

% \renewcommand{\thesection}{}			% Neudefinition der Section-Überschrift
% \renewcommand{\thesubsection}{\arabic{section}.\arabic{subsection}}	% Neudefinition der Subsection-Überschrift

\setlength{\textheight}{21cm}

\begin{document}
\hspace{0.87\textwidth}
\begin{minipage}{120pt}
\vspace{-1.8cm}
\includegraphics[width=80pt]{logo.pdf}
\centering
\small Zusammenkunft aller Physik-Fachschaften
\end{minipage}

\begin{center}
\vspace{1.5cm}
\huge{Bericht der SommerZaPF 2011 in Dresden} 
\vspace{1cm}
\end{center}

% \vspace{-7ex}
Vom 25.-29. Mai 2011 fand in Dresden die Zusammenkunft aller Physik-Fachschaften (ZaPF) statt. Die ZaPF ist die
Bundesfachschaftentagung der Physik und tagt einmal im Semester an Hochschulen im deutschsprachigen Raum, wobei sie von
der Physik-Fachschaft der ausrichtenden Hochschule selbst organisiert wird. Diesen Sommer hat die Fachschaft der TU
Dresden die ZaPF ausgetragen, an der Vertreter von 28 Fachschaften aus Deutschland teilnahmen.


\section*{Europäischer und Deutscher Qualifikationsrahmen (EQR/DQR)}
\vspace{-12pt}
Ein wichtiges Thema dieser ZaPF war der Europäische und Deutsche Qualifikationsrahmen zu dem es Ende 2011 einen
Beschluss im Bundestag geben soll. Der DQR hat das Ziel, das Gesamtkompetenzniveau eines Abschlusses in einer
einzelnen Zahl (Niveauindikator) greifbar zu machen. So soll das Niveau der Kompetenzen zweier Abschlüsse
mit unterschiedlichem Lehrinhalt vergleichbar gemacht werden.

Der Vergleich eines deutschen Abschlusses mit einem
Abschluss aus einem anderen EU-Land funktioniert über ein Verfahren, bei dem zuerst der Abgleich mit dem DQR, danach
mit dem EQR, dann mit den Nationalen Qualifikationsrahmen (NQR) des jeweiligen Landes und am Ende mit dem Abschluss 
selbst durchgeführt wurde.

Da dieses Konzept in hohem Maße auch die Studierenden der Physik betrifft, hat die ZaPF dazu eine
\href{http://zapfev.de/sites/default/files/2011_05_Stellungnahme_EQR-DQR_0.pdf}{Stellungnahme} verfasst, in der sie die
Vergleichbarkeit der Abschlüsse
zwischen verschiedenen Fachgebieten sowie auch innerhalb eines Fachgebietes anzweifelt, die mit dem DQR/EQR suggeriert
wird. Die ZaPF befürchtet außerdem, dass sich der DQR für Bewerbungen immer mehr zu einem verbindlichen Wert entwickelt.
Die Bologna-Reformen haben innerhalb des Hochschulsystems mit zweifelhaftem Erfolg einen Vergleich zwischen Abschlüssen
hervorrufen wollen, weswegen ein noch umfassenderer Vergleich über alle Ausbildungsgänge als zum Scheitern verurteilt
angesehen wird. 


\section*{ZEITLast Studie}
\vspace{-12pt}
Unter der Federführung von Prof. Rolf Schulmeister (Uni Hamburg) hat die Studie ZEITLast zum Ziel, die Arbeitsbelastung
von Studierenden verschiedener Bachelorstudiengänge zu ermitteln. Zu diesem Zweck werden Zeitbudget-Analysen
durchgeführt. Dabei füllen die Probanden täglich einen detailierten Online-Erfassungsbogen über ihren Tagesablauf aus,
was diese Studie im deutschsprachigen Raum bisher einzigartig macht.

In einer \href{http://zapfev.de/sites/default/files/2011_05_Stellungnahme_ZEITlast.pdf}{Stellungnahme} begrüßt die ZaPF
die Methodik der Studie ZEITLast zur protokollarischen Erhebung der Arbeitszeit,
insbesondere die zeitnahe Abfrage des Tagesablaufs der Studierenden mit einem detailreichen Erfassungsbogen. Sie
unterstützt die Ausweitung der Studie auf weitere Studiengänge, insbesondere der Physik. Für eine Erweiterung auf diesen
Fachbereich besteht seitens der ZaPF großes Interesse an einer engen Kooperation.

Die Arbeitsgrundlage des Arbeitskreises “Workload” auf der Sommer-ZaPF 2011 in Dresden war das Buch “Die Workload im
Bachelor: Zeitbudget und Studierverhalten” von Rolf Schulmeister und Christiane Metzger (Hrsg.), erschienen am
19.05.2011 bei Waxmann.

Inwiefern der Zeitaufwand mit der Arbeitsbelastung der Studierenden korreliert und welche weiteren Faktoren noch eine
Rolle spielen soll auf der nächsten ZaPF erörtert werden.

\section*{Föderalismus}
\vspace{-12pt}
Seit der WinterZaPF in München 2009 wurde auf den ZaPFen über den Föderalismus im Bildungssystem
diskutiert. Dazu soll
in Kürze eine Stellungnahme verfasst und veröffentlicht werden. Als Grundlage dafür wurde auf dieser ZaPF ein
\href{http://zapfev.de/sites/default/files/2011_05_Katalog-Foederalismus.pdf}{Katalog} beschlossen, in dem die
ZaPF folgende Probleme erkennt, die u.a.
durch die aktuelle föderale Struktur im Bildungswesen verursacht werden:
\begin{enumerate}
 \item Lehrerknappheit in Bundesländern soll durch Abwerbung aus anderen Bundesländern gelöst werden (Ursache:
Unkoordinierte Ausbildungs-, Einstellungs- und Lohnpolitik).
 \item Keine freie Wahl des Ausbildungsstandortes durch länderspezifische Vorgaben für Eintritt in den
Vorbereitungsdienst.
 \item Fehlende Vergleichbarkeit der Schul- und Hochschulabschlüsse zwischen den verschiedene Ländern (Ursache: Keine
einheitlichen Rahmengesetze).
 \item Keine Finanzierungsmöglichkeit von Bildungsangeboten durch den Bund erlaubt. Dadurch hängt die Bildungsqualität
von der Bildungsfinanzierungseinstellung der Bundesländern ab.
 \item Die unterschiedliche Erhebung von Studiengebühren führt zu einer verzerrten Studien\-orts- und Studienfachwahl.
 \item Mehrfache Verwaltungs- und Kontrollstrukturen im Bildungswesen führen zu Mehrkosten
\end{enumerate}
Deshalb begrüßt die ZaPF Bestrebungen die föderalen Strukturen im Bildungswesen zu überdenken.

\section*{Weitere Themen}
\vspace{-12pt}
Im Laufe der ZaPF wurden in verschiedenen Arbeitskreisen (AKs) noch viele weitere Themen diskutiert und Erfahrungen
ausgetauscht. So wurde nach der Vorarbeit auf den letzten beiden ZaPFen im Arbeitskreis \textbf{Übungskonzepte}
eine \href{http://zapfev.de/sites/default/files/2011_05_Ausarbeitung_\%C3\%9Cbungskonzepte.pdf}{Ausarbeitung} für
Übungskonzepte im Physikstudium verabschiedet.
Sie basiert auf dem Richtlinienkatalog für den Übungsbetrieb, welcher auf der WinterZaPF 2010 beschlossen wurde.

Auch dieses mal wurde das Thema \textbf{CHE-Ranking} behandelt, in dessen AK über die weitere Zusammenarbeit mit dem CHE
gesprochen wurde. Viele Fachschaften hatten sich inzwischen Detailauswertungen schicken lassen und ein erster Vergleich
zeigte
eine geringe Rücklaufquote, die in den meisten Fällen unter 25\,\% lag. Im Herbst sollen die neuen Fragebögen den
Fachschaften zur Verfügung gestellt werden, damit sie von diesen kommentiert werden können. Die Fachschaften sind
aufgerufen dieses zu tun und ihre Kommentare auch in das ZaPF-Wiki einzutragen.\\
Es wurde noch überlegt wie die Ampeldarstellung im ZEIT-Studienführer verbessert werden könnte, zum Beispiel durch die
Angabe von Noten mit Standardabweichung. Der AK wird voraussichtlich auf der nächsten ZaPF weitergeführt werden.

Im Arbeitskreis \textbf{Höhere Mathematik} wurde über die Verwendung von GTR/CAS-Taschen\-rechner im Schuluntericht
diskutiert
und deren Auswirkungen auf die mathematische Kompetenzen der Studienanfänger. Es wurde beobachtet, dass viele
Studierende anfangs Probleme mit den Konzepten einfacher Rechnungen haben, welche eventuell durch die
Nutzung solcher Rechner bedingt sein könnten. Da dieses Problem auch andere Fachrichtungen betreffen könnte, bittet die
ZaPF die anderen Bundesfachschaftentagungen (BuFaTas) sich mit diesem Thema auseinander zu setzen und die Ergebnisse zu
veröffentlichen. Ein Austausch von Erfahrungen zwischen den BuFaTas könnte eine größere Diskussionsgrundlage schaffen
und damit eine fundiertere Stellungnahme ermöglichen.

In Fortführung zur letzten ZaPF wurde in einem Arbeitskreis weiter über das \textbf{Selbstverständnis} der ZaPF
diskutiert. Fragestellung sind dabei wen die ZaPF vertritt und was ihre Aufgaben und Kompetenzen sind. Der Arbeitskreis
strebt dabei eine Überarbeitung der Satzung an und wird zu diesem Zweck auf der nächsten ZaPF fortgeführt.

Thema eines weiteren Arbeitskreises der ZaPF war die Diskussion über die \textbf{Prüfungslast} in
Bachelor/Master-Studengängen. Die Belastung wurde als zu groß beurteilt jedoch blieb eine Diskussion über
Verbesserungsmöglichkeiten weitestgehend konsenslos. Dies liegt hauptsächlich an den sehr
unterschliedlichen Studienplänen der Universitäten. So werden modulübergreifende Prüfungen oder unbenotete
Module sehr kontrovers angesehen. Während der Diskussion im Arbeitskreis kristallisierte sich jedoch eine konsensfähige
Rahmenbedingung heraus: Falls modulübergreifende Prüfungen abgelegt werden sollen, dann soll dies einzig in mündlicher
Form stattfinden. Desweiteren wurde festgestellt, dass semesterbegleitende Prüfungsleistungen, die für die Abschlussnote
relevant sind, die Studierenden stärker belasten. 

Im Arbeitskreis \textbf{Doppelabiturjahrgang} ging es um die Problematik des zu erwartenden starken Anstiegs von
Studienanfängern auf Grund der Einführung des G8 in vielen großen Bundesländern. Zum einen wurde dabei von den bereits
betroffenen Fachschaften berichtet, zum anderen wurde ein
\href{https://vmp.ethz.ch/zapfwiki/index.php/AK_Doppeljahrgang}{Katalog} erstellt, der mögliche Fragen sammelt, die
Studienanfänger an die Fachschaften herantragen könnten. Außerdem wurde ein weiterer
\href{https://vmp.ethz.ch/zapfwiki/index.php/AK_Doppeljahrgang}{Katalog} mit Fragen erstellt, mit denen sich die
Fachschaftsvertreter dringend auseinandersetzen sollten. Dazu werden teilweise bereits Lösungen vorgeschlagen. Die ZaPF
würde diese Liste gerne fachübergreifend ergänzen, da es sich hierbei um ein Thema handelt, dass innerhalb der nächsten
drei Jahre einen Großteil der deutschen Hochschulen und viele Fachbereiche betreffen wird.


% Der AK \textbf{Lehramt} hat sich auch diese ZaPF wieder mit Themen rund um das Lehramt beschäftigt.
% Unter anderem hat er sich mit der Stellungnahme der DPG auseinandergesetzt, welche sich auf die
% \href{http://zapfev.de/sites/default/files/Lehramtstellungnahme.pdf}{Stellungnahme} der ZaPF vom 09. Mai 2010 bezieht.
% Auch die Umfrage unter den Lehramtsstudierenden wurde besprochen und soll noch einmal überarbeitet werden,
% bevor sie an die Fachschaften verschickt wird.

Im AK \textbf{Berufungskommission} wurde über das Thema Lehrproben diskutiert. Die ZaPF hatte diese in einer
Resolution auf der WinterZaPF 2009 für alle Berufungsverfahren gefordert. Es
wurden die verschiedenen Konzepte an den Universitäten vorgestellt und sich auf ein paar grundlegende Punkte
geeinigt die bei der Umsetzung beachtet werden sollten. Außerdem wurde diskutiert, wie mit Anfragen anderer
Fachschaften bezüglich Berufungskadidaten umgegangen werden sollte. Aus Datenschutzgründen dürfen solche Anfragen
nicht gestellt werden, worauf im Falle einer entsprechenden Kontaktaufnahme hingewiesen werden muss.

Weitere Arbeitskreise haben sich mit vielen anderen Themen auseinander gesetzt, wie zum Beispiel
\textbf{Gleichstellung} und welche Probleme speziell in Grundpraktika auftreten. Es wurde auch weiter am
\textbf{Studienführer Physik} gearbeitet, der in nächster Zeit veröffentlicht werden soll. Aus aktuellem Anlass wurde
sich über \textbf{Verfasste Studierendenschaften} ausgetauscht, da diese in naher Zukunft in Baden-Württenberg wieder
eingeführt werden sollen. Auch zum \textbf{Lehramt} gab es wieder einen AK, der sich mit der Stellungnahme der DPG
auseinander gesetzt hat und eine weitere Umfrage unter den Lehrern plant.

Die genauen Protokolle der einzelnen Arbeitskreise sowie der beiden Plenen sind im ZaPF-
Wiki zufinden unter: \url{https://vmp.ethz.ch/zapfwiki/index.php/SoSe11}

\vspace{0.5cm}
Die nächste ZaPF findet vom 24. bis 27. November 2011 in \href{http://zapfibo.de}{Bonn} statt und wir hoffen euch alle
dort begrüßen zu können. 


\end{document}
