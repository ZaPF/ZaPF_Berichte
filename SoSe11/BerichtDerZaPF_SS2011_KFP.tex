\documentclass{scrartcl}
\usepackage[utf8]{inputenc}
\usepackage[T1]{fontenc}
\usepackage[ngerman]{babel}
\usepackage{graphicx}

\usepackage{fourier} 			% Schriftart
\usepackage[scaled=0.81]{helvet} 	% Schriftart
\usepackage{setspace}			% Zeilenabstand

\usepackage{url}
% \usepackage{tocloft} 			% Paket für Table of Contents

\usepackage{xcolor}
% \definecolor{tocblue}{HTML}{000033}
\definecolor{urlred}{HTML}{660000}

\usepackage{hyperref}
\hypersetup{
  colorlinks=true,	
  linkcolor=black,	% Farbe der internen Links (u.a. Table of Contents)
  urlcolor=urlred}	% Farbe der url-links

\parindent 0pt 				% Absatzeinrücken verhindern
\parskip 12pt 				% Absätze durch Lücke trennen

\pagestyle{empty}
% \setlength{\textheight}{23cm}

% \renewcommand{\thesection}{}			% Neudefinition der Section-Überschrift
% \renewcommand{\thesubsection}{\arabic{section}.\arabic{subsection}}	% Neudefinition der Subsection-Überschrift

\begin{document}

% \hspace{0.87\textwidth}
% \begin{minipage}{120pt}
% \vspace{-1.8cm}
% \includegraphics[width=80pt]{logo.png}
% \centering
% \small Zusammenkunft aller Physik-Fachschaften
% \end{minipage}
% 
% \begin{center}
% \vspace{1.5cm}
% \huge{Bericht der SommerZaPF 2011 in Dresden} 
% \vspace{1cm}
% \end{center}

\begin{minipage}{0.7\textwidth}
 \Huge{Bericht der SommerZaPF\\ 2011 in Dresden} 
\end{minipage}
\hfill
\begin{minipage}{120pt}
\vspace{-1cm}
\includegraphics[width=80pt]{logo.png}
\centering
\small Zusammenkunft aller Physik-Fachschaften
\end{minipage}
\vspace{7ex}

Vom 25.-29. Mai 2011 fand in Dresden die Zusammenkunft aller Physik-Fachschaften (ZaPF) statt. Die ZaPF ist die
Bundesfachschaftentagung der Physik und tagt einmal im Semester an Hochschulen im deutschsprachigen Raum, wobei sie von
der Physik-Fachschaft der ausrichtenden Hochschule selbst organisiert wird. Diesen Sommer hat die Fachschaft der TU
Dresden die ZaPF ausgetragen, an der Vertreter von 28 Fachschaften aus Deutschland teilnahmen.


\section*{Europäischer und Deutscher Qualifikationsrahmen (EQR/DQR)}
\vspace{-12pt}
Ein wichtiges Thema dieser ZaPF war der Europäische und Deutsche Qualifikationsrahmen zu dem es Ende 2011 einen
Beschluss im Bundestag geben soll. Der DQR hat das Ziel, das Gesamtkompetenzniveau eines Abschlusses in einer
einzelnen Zahl (Niveauindikator) greifbar zu machen. So soll das Niveau der Kompetenzen zweier Abschlüsse
mit unterschiedlichem Lehrinhalt vergleichbar gemacht werden.

Da dieses Konzept in hohem Maße auch die Studierenden der Physik betrifft, hat die ZaPF dazu eine
\href{http://zapfev.de/sites/default/files/2011_05_Stellungnahme_EQR-DQR_0.pdf}{Stellungnahme} verfasst, in der sie die
Vergleichbarkeit der Abschlüsse
zwischen verschiedenen Fachgebieten sowie auch innerhalb eines Fachgebietes anzweifelt, die mit dem DQR/EQR suggeriert
wird. Die ZaPF befürchtet außerdem, dass sich der DQR für Bewerbungen immer mehr zu einem verbindlichen Wert entwickelt.
Die Bologna-Reformen haben innerhalb des Hochschulsystems mit zweifelhaftem Erfolg einen Vergleich zwischen Abschlüssen
hervorrufen wollen, weswegen ein noch umfassenderer Vergleich über alle Ausbildungsgänge als zum Scheitern verurteilt
angesehen wird. 


\section*{ZEITLast Studie}
\vspace{-12pt}
Unter der Federführung von Prof. Rolf Schulmeister (Uni Hamburg) hat die Studie ZEITLast zum Ziel, die Arbeitsbelastung
von Studierenden verschiedener Bachelorstudiengänge zu ermitteln. Zu diesem Zweck werden Zeitbudget-Analysen
durchgeführt. Dabei füllen die Probanden täglich einen detailierten Online-Erfassungsbogen über ihren Tagesablauf aus,
was diese Studie im deutschsprachigen Raum bisher einzigartig macht.

In einer \href{http://zapfev.de/sites/default/files/2011_05_Stellungnahme_ZEITlast.pdf}{Stellungnahme} begrüßt die ZaPF
die Methodik der Studie ZEITLast zur protokollarischen Erhebung der Arbeitszeit,
insbesondere die zeitnahe Abfrage des Tagesablaufs der Studierenden mit einem detailreichen Erfassungsbogen. Sie
unterstützt die Ausweitung der Studie auf weitere Studiengänge, insbesondere der Physik. Für eine Erweiterung auf diesen
Fachbereich besteht seitens der ZaPF großes Interesse an einer engen Kooperation.

Die Arbeitsgrundlage war das Buch “Die Workload im
Bachelor: Zeitbudget und Studierverhalten” von Rolf Schulmeister und Christiane Metzger (Hrsg.), erschienen am
19.05.2011 bei Waxmann.

Inwiefern der Zeitaufwand mit der Arbeitsbelastung der Studierenden korreliert und welche weiteren Faktoren noch eine
Rolle spielen soll auf der nächsten ZaPF erörtert werden.

\section*{Weitere Themen}
\vspace{-12pt}
Auch dieses mal wurde das Thema \textbf{CHE-Ranking} behandelt, in dessen Arbeitskreis über die weitere Zusammenarbeit
mit dem CHE
gesprochen wurde. Viele Fachschaften hatten sich inzwischen Detailauswertungen schicken lassen und ein erster Vergleich
zeigte eine geringe Rücklaufquote, die in den meisten Fällen unter 25\,\% lag. Im Herbst sollen die neuen Fragebögen den
Fachschaften zur Verfügung gestellt werden, damit sie von diesen kommentiert werden können. Es wurde noch überlegt wie
die Ampeldarstellung im ZEIT-Studienführer verbessert werden könnte, zum Beispiel durch die
Angabe von Noten mit Standardabweichung.

Der Arbeitskreis \textbf{Lehramt} hat sich diese ZaPF unter anderem mit der Stellungnahme des Fachverbandes
\textit{Didaktik der Physik} der DPG auseinandergesetzt, welche sich auf die
\href{http://zapfev.de/sites/default/files/Lehramtstellungnahme.pdf}{Stellungnahme} der ZaPF vom 09. Mai 2010 bezieht.
Auch die Umfrage unter den Lehramtsstudierenden wurde besprochen und soll noch einmal überarbeitet werden,
bevor sie an die Fachschaften verschickt wird.

Im Arbeitskreis \textbf{Übungskonzepte} wurde eine
\href{http://zapfev.de/sites/default/files/2011_05_Ausarbeitung_\%C3\%9Cbungskonzepte.pdf}{Ausarbeitung} für
Übungskonzepte im Physikstudium verabschiedet. Es wurde außerdem weiter am \textbf{Studienführer Physik} gearbeitet,
der in nächster Zeit veröffentlicht werden soll. Dieser stellt kein weiteres Hochschul-Ranking dar, sondern soll
vorallem die allgemeinen Studienbedingungen (z.B. Wohnplätze) aufzeigen. Aus aktuellem Hintergrund war auch die
Problematik der \textbf{Doppelabiturjahrgängen} ein wichtiges Thema. Dabei ist deutlich geworden, dass der Anstieg der
Bewerberzahlen an einigen Fachbereichen unterschätzt wurden.
 
\end{document}
